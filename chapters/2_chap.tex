%LTeX: language=de-DE
\chapter{Mathematische Grundlagen}
    Um sich dem eingangs geschilderten Problem adäquat zu nähern, sollen zunächst die mathematischen Grundlagen expliziert werden.
    \section*{Kontinuität und Massenbilanz}
        Aus dem Erhaltungssatz der Masse folgt, dass zu jedem Zeitpunkt und durch jeden Querschnitt die gleiche Masse angesaugt
        wie ausgeströmt wird. Mit entgegen gesetzten Vorzeichen für ein- und austretende Massen ergibt sich:
        \begin{equation}
            \frac{dM}{dt} = \dot{m_1} - \dot{m_2} = 0 \qquad \Leftrightarrow \qquad \dot{m_1} = \dot{m_2}
            \label{eq:massenbilanz}
        \end{equation}

        Hierbei ist die zeitliche Änderung der Einzelmasse abhängig der Dichte \(\rho\) des Fluids, der durchströmten Fläche
        \(A\) und dem Linienelement entlang eines Stromfadens \(s\). Dies führt zu
        \begin{equation}
            \dot{m} = \rho \dot{V} = \rho A \frac{\partial s}{\partial t} \qquad \text{mit} \qquad \frac{\partial s}{\partial t} = c
            \label{eq:mass flow}
        \end{equation}
        Mit \(\frac{\partial s}{\partial t}\), was bekanntlich einer Geschwindigkeit entspricht, der zusätzlichen Annahme,
        dass \(\rho_1 = \rho_2\) -- die Dichte sich im Medium also nicht verändert -- und \cref{eq:mass flow} eingesetzt in \cref{eq:massenbilanz}
        lässt sich
        \begin{equation}
            c_1 A_1 = c_2 A_2 = konst
            \label{eq:kontinuitaetsgleichung volumen}
        \end{equation}
        aufschreiben. \Cref{eq:kontinuitaetsgleichung volumen} ist die Kontinuitätsgleichung des Volumens.
        \subsection*{\textsc{Reynolds}-Zahl}
        Die dimensionslose \textsc{Reynolds}-Zahl ist ein Maß für die \enquote{Neigung} eines Mediums zu turbulentem Verhalten üblicherweise
        als Funktion der Strömungsgeschwindigkeit und der um- oder durchströmten Geometrie.
        \begin{equation}
            Re = \frac{c \cdot d_h \cdot \rho}{\nu}
            \label{eq:reynold}
        \end{equation}
        mit der charakteristischen Länge \(d_h\), der dynamischen Viskosität \(\nu\), Dichte \(\rho\) und der Geschwindigkeit entlang des Stromfadens \(c\).
        Dichte und dynamische Viskosität sind materialabhängig während \(d_h\) die Geometrie widerspiegelt.
        Unterhalb einer kritischen \textsc{Reynolds}-Zahl \(Re_{krit}\) von \(2320\) kann von laminarer Strömung ausgegangen werden \cite{Hans.Witt.Lueftungstechnik.2021}.
        \par\medskip
        \section*{\textsc{Bernoulli}}
        \Cref{eq:bernoulli druck} ist die allgemeine Form \textsc{Bernoullis} Druckgleichung.
        \begin{equation}
            \frac{1}{2} \rho_1 c_1^2 + \rho_1 gh_1 + P_1 = \frac{1}{2} \rho_2 c_2^2 + \rho_2 gh_2 + P_2 + \Delta P_R + \Delta P_F
            \label{eq:bernoulli druck}
        \end{equation}
        Unter der Annahme, dass Druckverluste durch Reibung vernachlässigt werden können, \(h_1 = h_2\) gilt und die Dichte
        \(\rho\) an jeder Stelle gleich ist, kann obige Gleichung vereinfacht aufgeschrieben werden als
        \begin{equation}
            \frac{1}{2} \rho c_1^2 + P_1 = \frac{1}{2} \rho c_2^2 + P_2 + \Delta P_F
            \label{eq:bernoulli energiegleichung vereinfacht}
        \end{equation}

        Die ersten beiden Terme in \cref{eq:bernoulli energiegleichung vereinfacht} geben den dynamischen Druck \(P_d\) und den statischen Druck \(P_s\)
        jeweils an der Stelle 1. Der dritte und vierte Term respektive die jeweiligen Drücke an der Stelle 2. Der letzte Term
        bildet den Druckverlust des Systems bedingt durch geometrische Gegebenheiten ab.\par
        \begin{equation}
            \Delta P_{F} = \zeta \cdot \frac{1}{2}\rho c^2 = \zeta \Delta P_{d}
            \label{eq:zeta}
        \end{equation}
        Der dimensionslose Faktor \(\zeta\) ist das Verhältnis aus formbedingtem Druckverlust entlang der Geometrie zu mittlerem
        dynamischen Druck unmittelbar nach der druckwiderstandbehafteten Geometrie \cite{Grundlagen.der.Ventilatorentechnik.2021}.
        \Cref{eq:bernoulli energiegleichung vereinfacht,eq:zeta} ergeben
        \begin{equation}
            \frac{1}{2} \rho c_1^2 + P_1 = \frac{1}{2} \rho c_2^2 + P_2 + \sum_i \zeta_i \Delta P_{d_i}
        \end{equation}
        wobei die Indices \(i\) alle widerstandsbehafteten Teilstrecken zwischen den Stellen \(1\) und \(2\) sind.

        \nocite{Durst.2006.Grundlagen.der.Stroemungsmechanik,Surek.2017.Technische.Stroemungsmechanik}