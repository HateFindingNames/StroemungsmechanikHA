%LTeX: language=de-DE
\chapter{Problemstellung}
    % Wichtige Kenngrößen zu im Handel erhältlichen Lüftern sind unter Anderem Volumenstrom und statischer Druck. Ersteres
    % ist ein Maß für die je Zeiteinheit geförderte Luftmenge und wird häufig in \(\frac{m^3}{h}\) angegeben. Letzteres
    % gibt den 

    Massenbilanz: angenommen wird, dass zu jedem Zeitpunkt die gleiche Masse angesaugt wie ausgeströmt wird. Diese Annahme
    resultiert aus zu erwartenden Mach-Zahlen deutlich unter \( 0,3 \) und damit verbundener zu vernachlässigend geringer
    Dichteänderung des Mediums. Mit entgegen gesetzten Vorzeichen für ein- und austretende Massen ergibt sich:
    \begin{equation}
        \frac{dM}{dt} = \dot{m_1} + \dot{m_2} = 0
    \end{equation}

    Hierbei ist die zeitliche Änderung der Einzelmasse abhängig der Dichte \(\rho\) des Fluids, der durchströmten Fläche
    \(A\) und dem Linienelement entlang eines Stromfadens \(s\). Dies führt zu
    \begin{equation}
        \dot{m} = \rho A \frac{\partial s}{\partial t} \qquad \text{mit} \qquad \frac{\partial s}{\partial t} = c
    \end{equation}

    \textbf{Reynolds-Zahl}
    \begin{equation}
        Re = \frac{\nu}{c \cdot d}
    \end{equation}
    mit der charakteristischen Länge \(d\), der dynamischen Viskosität \(\nu\) und der geschwindigkeit entlang des Stromfadens \(c\).
    \par\medskip
    \textbf{Bernoulli} (Energieform)
    \begin{equation}
        \frac{1}{2} \rho_1 c_1^2 + \rho_1 gh_1 + P_1 = \frac{1}{2} \rho_2 c_2^2 + \rho_2 gh_2 + P_2 + \Delta P_R + \Delta P_F
    \end{equation}
    bzw. in vereinfachter Form mit der Annahme, dass Druckverluste durch Reibung vernachlässigt werden können, \(h_1 = h_2\) gilt
    und die Dichte \(\rho\) an jeder Stelle gleich ist
    \begin{equation}
        \frac{1}{2} \rho c_1^2 + P_1 = \frac{1}{2} \rho c_2^2 + P_2 + \Delta P_F
        \label{eq:bernoulli energiegleichung vereinfacht}
    \end{equation}

    Die ersten beiden Terme in \cref{eq:bernoulli energiegleichung vereinfacht} geben den dynamischen Druck \(P_d\) und den statischen Druck \(P_s\)
    jeweils an der Stelle 1. Der dritte und vierte Term respektive die jeweiligen Drücke an der Stelle 2. Der letzte Term
    bildet den Druckverlust des Systems bedingt durch geometrische Gegebenheiten ab.\par
    \begin{equation}
        \Delta P_F = \zeta \cdot \frac{1}{2}\rho c_2^2
    \end{equation}
    \par\medskip
    \textbf{Kontinuität}
    \begin{equation}
        \rho_1 c_1 A_1 = \rho_2 c_2 A_2 = \dot{m} = konst.
    \end{equation}
    \par\medskip
    Was zum \textbf{Volumenstrom} führt
    \begin{equation}
        \dot{V_i} = c_i \cdot A_i
    \end{equation}
    \par\medskip
    \textbf{Navier-Stokes}
    \begin{equation}
        -\vec{\nabla} P + \eta\vec{\nabla}^2\vec{c} + \rho\vec{f} = \rho \left(\frac{\partial\vec{c}}{\partial t} + \left(\vec{c} \cdot \vec{\nabla}\right) \cdot \vec{c}\right)
    \end{equation}
    Vereinfachte Form mit
    \begin{itemize}
        \item Statisches Fluid: \(\frac{\partial \vec{c}}{\partial t} = \vec{0}\)
        \item Fehlen äußerer Kräfte: \(\vec{f} = \vec{0}\)
        \item Vernachlässigung innerer Reibung: \(\eta\vec{\nabla}^2\vec{c} = \vec{0}\)
    \end{itemize}
    \medskip
    \begin{equation}
        -\vec{\nabla}P = \rho \left(\vec{c} \cdot \vec{\nabla}\right) \cdot \vec{c}
    \end{equation}