%LTeX: language=de-DE
\chapter{Problemstellung}
    In vielen technischen Anwendungen ist eine ausreichende Bauteilkühlung unerlässlich, um einen störungsfreien Betrieb
    und lange Lebensdauern der Bauteile zu gewährleisten. Nicht immer lassen sich jedoch Kompressoren oder Gebläse im industriellen
    Maßstab einsetzen -- sei es aus Kostengründen oder durch möglichst kompakte Bauformen.

    Im Folgenden soll untersucht werden, inwiefern sich die Performanz eines gegebenen Ventilators zum Betrieb in klein- 
    und mittelgroßen Geräten bezüglich der für den Einsatz wichtigen Größen \textit{Volumenstrom} und \textit{statischer Druck}
    auf die jeweilige Arbeitsumgebung hin optimieren lässt.