%LTeX: language=de-DE
\chapter{Steckbrief}
    \section{Einordnung}
        Acrylnitril-Butadien-Styrol oder kurz ABS ist ein Terpolymer, dass zur Gruppe der (amorphen) Thermoplasten gehört. Aufgrund
        der chemischen Struktur ist eine weitere Unterordnung in die \textit{\enquote{hochschlagfesten Styrol-Copolymere}}\cite{Eyerer.2020.Polymer.Engineering.1}
        möglich.\par
        Weitere bekannte Vertreter der amorphen Thermoplaste sind das Polyvinylchlorid (PVC), Polystyrol (PS), Polycarbonat (PC)
        oder Polymethylmethacrylat (PMMA). Ihnen allen gemein ist eine gute bis sehr gute optische Transmisivität bedingt durch ihren amorphen Aufbau.
        %
        \subsection{Varianten und Vergleich}
            Die folgenden Diagramme sollen einen Überblick über die wichtigsten Schlüsseleigenschaften des ABS im direkten Vergleich
            mit seinen Verwandten Derivaten und Polycarbonat als Referenz liefern. Werte wurden jeweils auf das in seiner
            Kategorie am besten abschneidende Material normiert und bilden damit relative Größen ab. Dies erschien sinnvoll,
            da es einen Vergleich \enquote{auf einen Blick} zulässt.\par
            \begin{figure}[H]
                \centering
                \includesvg[inkscapelatex=false, width=\textwidth]{RadarCharts/mechanical/mechanical.svg}
                \caption{Mechanisches Profil.}
                \label{fig:pc mechanical profile}
            \end{figure}\par
            %
            \Cref{fig:pc mechanical profile} zeigt die mechanische Performanz des ABS\@. Neben anderen zur Auswahl stehenden
            Metriken wurde sich hier der Übersicht halber auf die Vergleichsgrößen Reißdehnung, Kugeldruckhärte, Biegefestigkeit
            und im Falle der Kunststoffe insbesondere dem Kriechmodul beschränkt.

            Es ist zu erkennen, dass ABS hier selbst im Vergleich zu seinen Derivaten unterwältigt. Einzig das
            Acrylnitril-Styrol-Acrylat (ASA) liegt hier in etwa gleichauf. Die thermischen und insbesondere die chemischen Profile
            (\cref{fig:pc thermal profile} und \cref{fig:pc chemical profile}) geben eine leicht differenzierteres Bild.
            \begin{figure}[H]
                \centering
                \includesvg[inkscapelatex=false, width=\textwidth]{RadarCharts/thermal/thermal.svg}
                \caption{Thermisches Profil.}
                \label{fig:pc thermal profile}
            \end{figure}
            %
            Auch thermisch unterliegt überwiegend das ABS\@. Es zeigt sich, dass die Stärke des Materials in seinem chemischen
            Profil verortet zu sein scheint.\par
            Hier sei an dieser Stelle angemerkt, dass ASA als eine Weiterentwicklung des ABS zu verstehen ist (siehe \cref{sec:geschichte}).
            So überrascht es wenig, dass dem gegenüber ABS in vielen Punkten unterliegt.
            \begin{figure}[H]
                \centering
                \includesvg[inkscapelatex=false, width=\textwidth]{RadarCharts/chemical/chemical.svg}
                \caption{Chemisches Profil.}
                \label{fig:pc chemical profile}
            \end{figure}
            \nocite{datenblattsammlung.KERN.20210201}
        \subsection{Preis}
        %
        In Westeuropa wird ABS derzeit – unabhängig davon, ob es sich um regranuliertes oder \textit{virgin}\footnote{Dt.\ jungfräulich – Material zum Erstgebrauch.}
        ABS handelt – zwischen ca. €0,85 und €3,00 je Kilogram gehandelt.
    \section{Strukturformel}
        \begin{figure}[H]%
            \centering
            \adjustbox{max width=\textwidth}{
                \subfloat[Acrylnitril.\label{subfig:structural formula acrylonitril}]{\includesvg[inkscapelatex=false, scale=.5]{zeichnungen/acrylnitril.svg}}%
            }
            \qquad
            \adjustbox{max width=\textwidth}{
                \subfloat[Styrol.\label{subfig:structural formula styrene}]{\includesvg[inkscapelatex=false, scale=.5]{zeichnungen/styrol.svg}}%
            }
            \qquad
            \adjustbox{max width=\textwidth}{
                \subfloat[1,3-Butadien.\label{subfig:structural formula butadiene}]{\includesvg[inkscapelatex=false, scale=.5]{zeichnungen/butadien.svg}}%
            }
                \caption[Strukturformeln der monomeren Bestandteile des ABS]{Strukturformeln der monomeren Bestandteile des ABS.}%
            \label{fig:strukturformeln monomere}%
        \end{figure}
        %
        \begin{figure}[H]%
            \centering
            \hfill
            \adjustbox{max width=\linewidth, warn width}{
                \subfloat[Polyacrylnitril.\label{subfig:structural formula polyacrylonitril}]{\includesvg[inkscape=true, inkscapelatex=false, scale=.5]{zeichnungen/polyacrylnitril.svg}}%
            }
            \qquad
            % \hfill
            \centering
            \adjustbox{max width=\linewidth}{
                \subfloat[Polystyrol.\label{subfig:structural formula polystyrene}]{\includesvg[inkscape=true, inkscapelatex=false, scale=.5]{zeichnungen/polystyrol.svg}}%
            }
            \qquad
            % \hfill
            \centering
            \adjustbox{max width=\linewidth}{
                \subfloat[Polybutadien.\label{subfig:structural formula polybutadiene}]{\includesvg[inkscape=true, inkscapelatex=false, scale=.5]{zeichnungen/polybutadien.svg}}%
            }
                \caption[Strukturformeln der polymeren Bestandteile des ABS]{Strukturformeln der polymeren Bestandteile des ABS.}%
            \label{fig:strukturformeln polymere}%
        \end{figure}
    \section{Herstellung}
        ABS besteht aus den drei Monomeren Acrylnitril, Butadien und Styrol. Die folgenden Abschnitte sollen einen groben
        Überblick zur Synthese der jeweiligen Monomere verschaffen, um darauf aufbauend Verfahren zur Polymerisation aufzuzeigen.
        %
        \subsection{Acrylnitril}
            Eine Möglichkeit der Synthese von Acrylnitril, die technisch heute noch in großem Maßstab umgesetzt wird, ist das nach dem gleichnamigen
            Unternehmen benannte \textsc{Sohio}-Verfahren (siehe \cref{sec:geschichte}).\par
            Die Ausgangsstoffe sind hier Propen, (Luft-)Sauerstoff und Ammoniak. In Gegenwart eines mineralischen Katalysators wird der Wasserstoff
            des einfach gebundenen Kohlenstoffatoms abgespalten und der Stickstoff des Ammoniaks lagert sich an. Die Edukte der
            stark exothermen Reaktion sind das gewünschte Acrylnitril und Wasser \cite{sohio.process.patent.1959.9201957}.
        \subsection{Butadien}
            Das Monomer 1,3-Butadien findet in polymerisierter Form (Polybutadien) überwiegend in der Autoreifenindustrie als synthetischer Gummi
            Anwendung. Es ist ein Nebenprodukt bei der Aufspaltung längerkettiger Kohlenwasserstoffe meist fossilen Ursprungs.
        \subsection{Styrol}
            Eines der heute wichtigsten Verfahren zur Produktion von Styrol ist die Dehydrierung von Ethylbenzol
            an einem mineralischen Katalysator in Gegenwart von Wasserdampf unter hohem Druck \cite{styrol.synthese.Liquid-Phase.Alkylation.of.Benzene.Bellussi.1995}.\par
        \subsection{Polymerisation}
            \begin{itemize}
                \item Pfropfpolymerisation von Styrol und Acrylnitril auf einen vorgelegten Polybutadienlatex; das erhaltene Pfropfpolymerisat
                wird mit einem getrennt hergestellten SAN-Latex abgemischt, koaguliert und getrocknet. \cite{Domininghaus.1998.Kunststoffe.und.ihre.Eigenschaften,Eyerer.2020.Polymer.Engineering.1}
                \item Pfropfpolymer und SAN werden getrennt hergestellt, isoliert und getrocknet, schließlich nach dem Abmischen granuliert. \cite{Domininghaus.1998.Kunststoffe.und.ihre.Eigenschaften,Eyerer.2020.Polymer.Engineering.1}
            \end{itemize}
    \section{Anwendung}
            Aus der Kategorie der Kunststoffe ist Acrylnitril-Butadien-Styrol das meist verwendete Material für Produkt und
            Ingenieursanwendungen.\par
            Aufgrund seiner Temperaturbeständigkeit und hohen Schlagfestigkeit findet es breite Anwendung insbesondere in der
            Automobilindustrie zur Fertigung von Interieur, als Gehäuseteile für Elektronikprodukte oder auch Spielzeug. So verdankt etwa der Hersteller
            \textsc{LEGO} seinen Siegeszug der besonderen Eigenschaftenkomposition des Copolymers. Die thermoplastische Komponente
            des in hohen Anteilen vorhandenen Polystyrols macht die Produktion einfach, schnell, günstig und damit geeignet
            für die Massenproduktion. Die durch das 1,3-Butadien verliehene Elastizität sorgt für Formbeständigkeit auch bei
            wiederholter Nutzung der \textsc{LEGO}-Steine.
    \section{Eigenschaften}
            Chemische und thermische Standfestigkeit durch Acrylnitril, Thermoplastisch verformbar durch Polystyrol, Zähigkeit
            und Schlagfestigkeit durch 1,3-Butadien.\par
            %
            Als Nitrilkautschuk ist unter anderem in Form von Nitrilhandschuhen ist das Copolymer aus Acrylnitril und
            1,3-Butadien gut bekannt. Der Butadienanteil verleiht dem Material seine Elastitzität
            %
            Mischungsverhältnisse dieser drei Komponenten beeinflussen die jeweils mit ihnen assoziierten Eigenschaften des
            Gesamtmaterials. Hierdurch lassen sich durch relativ einfache Weise eine Großzahl verschiedener Varianten zum
            spezialisierten Einsatz produzieren. Dies spiegelt sich in der Vielzahl der auf dem Markt erhältlichen und unter
            verschiedenen Handelsnamen vertriebenen Varianten wider.\par
            %
            Eine eher mäßige Wetterstandfestigkeit rührt vom Polybutadien-Rückrad her. Die konjugierte Doppelbindung des
            Butadiens wird durch einfallendes Licht im ultravioletten Spektralbereich angeregt. In Gegenwart von Sauerstoff
            werden Polymerketten unter Bildung von Hydroxyl- und Carboxylgruppen aufgespalten. Ultimativ führt dies zu
            frühzeitiger Alterung des Materials was sich als Verfärbung und Minderung der mechanischen Eigenschaften zeigt.
            Um dem entgegenzuwirken werden mitunter zwar Fotostabilisatoren und/oder Radikalfänger eingesetzt, in der 
            Regel jedoch wird das ABS um seine mechanischen Eigenschaften möglichst zu erhalten mit einem schützenden Lack
            überzogen oder galvanisch behandelt \cite{Thermal.and.Photo-Degradation.of.Unstabilized.ABS.Adeniyi.1984,Domininghaus.1998.Kunststoffe.und.ihre.Eigenschaften}.
    \section{Besonderheiten}