%LTeX: language=de-DE
\chapter{Diskusion und Reflexion}
    Die aus der Simulation erhaltenen Daten legen nahe, dass, um einen erforderlichen Volumenstrom aufrechtzuerhalten, es
    durchaus sinnvoll sein kann die Geometrie, in der die Druck erzeugende Maschine verbaut wird, mitzuberücksichtigen.

    Der Lüfter des Typs \textit{4414F} wurde gewählt, da er zusammen mit Kennlinie und CAD in der integrierten technischen Datenbank der verwendeten
    Software enthalten war. Wünschenswert wäre gewesen aus den der Lüfterkennlinie und den ermittelten Kennlinienfeldern
    den Arbeitspunkt des Lüfters zu ermitteln und darin weitere Simulationen durchzuführen. Der zeitliche Imperativ ließ dies
    allerdings nicht mehr zu.

    Zukünftig ist durchaus vorstellbar, ein kleineres, eigenes Programm zur Berechnung und/oder Simulation zu erstellen.
    Einerseits um das erworbene Verständnis zu vertiefen und, andererseits, um die Arbeitsweise/Algorithmen der verschiedenen FEM
    besser zu durchdringen\footnote{Durchaus auch aus einem sportlichen Gedanken heraus}.
