%LTeX: language=de-DE
\chapter{Beispielrechnung und Simulation}
    Zur Simulation wurde ein Lüfter des Herstellers \textit{Papst} aus der Serie \textit{4414F}\cite{Papst.4414F.Luefter.2021} mit einem Volumenstrom
    von \SI{0,0474}{\frac{m^3}{s}} (frei blasend) jeweils an einem Diffusor und einer Düse betrachtet. Geometrischer Parameter
    sind Öffnungswinkel und Verjüngungswinkel \(\alpha\) des Diffusors bzw. der Düse bezogen auf die Symmetrieachse.
    Alle Körper sind rotationssymmetrisch und haben eine Länge von \SI{400}{mm} mit einem Eingangsdurchmesser von \SI{120}{mm}
    (vgl. \cref{fig:zeichnung diffusor,fig:zeichnung duese}).

    Die verwendete CAD- und FEM-Software ist Solidworks des französischen Entwicklers \textit{Dassault Systèmes} und das integrierte
    Zusatzmodul \textit{Flow Simulation}\cite{numerical.basis.solidworks.flow.simulation.2021}.

    Randbedingungen aller Simulationen waren ideale -- reibungsfreie und adiabate -- Wände. Weiter soll der Volumenstrom durch
    die gesamte Eingangsfläche gegen einen Umgebungsdruck von \SI{101325}{Pa} ausblasen.
    
    Es wurden für die verschiedenen Geometrien jeweils zwei Simulationskampagnen mit verschiedenen Berechnungszielen durchgeführt.
    Die erste Kampagne sollte eine qualitative Aussage über das Strömungs- und Druckverhalten der verschiedenen Körper ermöglichen.
    Hierzu wurde jeweils ein Geschwindigkeits- und ein Druckprofil im Mittenschnitt der Körper erstellt.
    Aus \cref{eq:zeta} geht hervor, dass der Druckverlust \(\Delta P_F\) sich quadratisch-proportional zur Strömungsgeschwindigkeit verhält.
    Es lag also nahe, eine weitere Kampagne durchzuführen, um diesen Einfluss zu untersuchen.
    
    So wurde einerseits der Volumenstrom schrittweise von \SI{0}{\frac{m^3}{s}} bis \SI{0,5}{\frac{m^3}{s}} erhöht, als auch
    die Winkel von \SI{2}{\degree} bis \SI{10}{\degree} in Schritten von \SI{2}{\degree} für den Diffusor und von \SI{0,5}{\degree}
    bis \SI{-5}{\degree} in Schritten von \SI{0,5}{\degree} für die Düse variiert. Um den Effekt des Öffnungswinkels auf den
    Druckwiderstand etwas weiter zu unterstreichen wurde im Falle des Diffusors zusätzlich eine Simulation für einen deutlich
    zu groß gewählten Öffnungswinkel von \SI{25}{\degree} durchgeführt.

    Da sich der Querschnitt in allen Fällen kontinuierlich ändert, wurde für die Bezugsgeschwindigkeit \(c\) in \cref{eq:kontinuitaetsgleichung volumen}
    als Berechnungsziel die Globale mittlere Geschwindigkeit gewählt. Weiter wurden der Totaldruck jeweils an Eingang und
    Ausgang ermittelt, um aus der Differenz \(\Delta P\) zu bilden.

    \section{Diffusor}
        Ein Diffusor wird in vielen technischen Bereichen eingesetzt, um den Luftstrom zu verzögern und/oder Druck rück zu gewinnen.
        Düsentriebwerke etwa benötigen zum Betrieb Luftsauerstoff in der Brennkammer. Hier ist es von Nachteil, wenn es aufgrund von
        Strömungen im Überschallbereich zu Stoßwellen kommt. Schlimmstenfalls kann es zu permanenten Beschädigungen des Triebwerks kommen.
        Um dem zu entgegnen werden sich aufweitende Kavitäten nahe dem Kompressor eingesetzt, um das einströmende Medium auf Mach < 1 abzubremsen, bevor es
        in die Brennkammer eingeleitet wird. Doch auch in der Gebäude- oder Bauteilekühlung ist es mitunter von Vorteil etwa vor einem
        Kühlgitter für einen Druckrückgewinn zu sorgen, um den zur Kühlung notwendigen Volumenstrom aufrechtzuerhalten.
        \begin{figure}[h]
            \centering
            \includegraphics[width=.8\textwidth]{SW_sim/results/diff_2deg.jpg}
            \caption[Geschwindigkeits- und Druckprofil am Diffusor bei \(\alpha = \SI{2}{\degree}\)]{Gezeigt wird das Geschwindigkeitsprofil des Luftstromes (schwarze Linien) und der relative Druck bezogen auf den Umgebungsdruck (Farbstufen) entlang des Diffusors bei \(\alpha = \SI{2}{\degree}\).}
            \label{fig:diff_2deg profil}
        \end{figure}
        Um unumkehrbare Druck- und damit Energieverluste zu minimieren\footnote{Kinetische Energie in Verwirblungen ist quasi Ortsfest und damit für den Rest des Systems verloren.} ist allerdings darauf zu achten, dass es nicht zu Strömungswiderständen
        durch Verwirblungen kommt. Einschlägige Tabellenwerke geben hierbei einen kritischen Öffnungswinkel von \(\approx \SI{10}{\degree}\) an \cite{Hans.Witt.Lueftungstechnik.2021,Grundlagen.der.Ventilatorentechnik.2021}.
        Mit Blick auf \cref{fig:diff_10deg profil} zeigt sich, dass es nahe der Wandungen am Ausgang des betrachteten Diffusors
        ab einem Winkel von \SI{10}{\degree} in der Tat zu Ablösungen kommt.
        \begin{figure}[h]
            \centering
            \includegraphics[width=.8\textwidth]{SW_sim/results/diff_10deg.jpg}
            \caption[Geschwindigkeits- und Druckprofil am Diffusor bei \(\alpha = \SI{10}{\degree}\)]{Geschwindigkeitsprofil des Luftstromes (schwarze Linien) und der relative Druck bezogen auf den Umgebungsdruck (Farbstufen) entlang des Diffusors bei \(\alpha = \SI{10}{\degree}\). Es ist zu erkennen, dass sich nahe der Wandung entlang des Auslasses Ablösungen beginnen auszubilden.}
            \label{fig:diff_10deg profil}
        \end{figure}
        Der Zusammenhang zwischen Volumenstrom und Druckverlust am Diffusor ist in \cref{fig:anlagenkennlinienfeld Diffusor} dargestellt.
        Eine steilere Kurve bedeutet einen bei gleich bleibendem Volumenstrom höheren Druckverlust.
        \begin{figure}[h]
            \centering
            \includesvg[inkscapelatex=false, width=.7\linewidth]{scidavis/kennlinienfeldDiffusor.svg}
            \caption[Anlagenkennlinenfeld Diffusor]{Anlagenkennlinienfeld eines Diffusors unter Variation des Öffnungswinkels.}
            \label{fig:anlagenkennlinienfeld Diffusor}
        \end{figure}
        \Cref{fig:zetas Diffusor} trägt den Druckwiderstandsbeiwert als Funktion des Winkels auf. Auch hier ist der Zusammenhang
        zwischen dem Öffnungswinkel und dem Druckwiderstandsbeiwert \(\zeta\) deutlich zu erkennen.
        \begin{figure}[H]
            \centering
            \includesvg[inkscapelatex=false, width=.7\linewidth]{scidavis/zetasDiffusor.svg}
            \caption[Druckverlustbeiwert als Funktion des Öffnungswinkels eines Diffusors]{Verlauf des Druckverlustbeiwertes \(\zeta\) eines Diffusors als Funktion des Öffnungswinkels \(\alpha\).}
            \label{fig:zetas Diffusor}
        \end{figure}
        %
    \section{Düse}
        Bei unveränderten Randbedingungen und Zielparametern wurden die obigen Simulationen für einen negativen Öffnungswinkel
        -- also einem sich verjüngenden Querschnitt -- durch geführt. Nach \cref{eq:kontinuitaetsgleichung volumen} ist zu erwarten,
        dass sich die Strömungsgeschwindigkeit zu Ausgang hin erhöht und der Druck damit mindert.
        \begin{figure}[H]
            \centering
            \includegraphics[width=.8\textwidth]{SW_sim/results/duese_-1deg.jpg}
            \caption[Geschwindigkeits- und Druckprofil an einer Düsenform bei \(\alpha = \SI{-1}{\degree}\)]{Strömungs- und Druckprofil an einer Düsenform mit einem Winkel von \(\alpha \SI{-1}{\degree}\). Analog zu den Profilen des Diffusors stellt sich hier ein umgekehrtes Verhältnis von Druck zu Strömungsgeschwindigkeit ein. Die Farbskala stellt auch hier den Druck relativ zum Umgebungsdruck dar.}
            \label{fig:duese_-1deg profil}
        \end{figure}
        Dieses Verhalten ist in den Profilen aus \Cref{fig:duese_-1deg profil,fig:duese_-5deg profil} ersichtlich. Zu erkennen ist jedoch,
        dass es hierbei nicht zu Ablösungen und infolgedessen zu unumkehrbaren Druckverlusten kommt.
        \begin{figure}[H]
            \centering
            \includegraphics[width=.8\textwidth]{SW_sim/results/duese_-5deg.jpg}
            \caption[Geschwindigkeits- und Druckprofil an einer Düsenform bei \(\alpha = \SI{-5}{\degree}\)]{Weitere verjüngung der Düsenform mit \(\alpha = \SI{-5}{\degree}\).}
            \label{fig:duese_-5deg profil}
        \end{figure}
        \Cref{fig:anlagenkennlinienfeld Duese} zeigt das Kennlinienfeld der unterschiedlichen Düsenwinkel. Auch hier ist ein plötzlicher
        Anstieg ab einem kritischen Winkel zu erkennen. Hervorzuheben ist allerdings der deutlich stärkere Einfluss des Winkels
        auf den Druckverlust. \Cref{fig:zetas Duese} trägt die zugehörigen Werte für \(\zeta\) über dem Winkel auf. Hier ist anzumerken,
        dass die Varianz für \(\zeta\) bei einem Winkel von \SI{-4,5}{\degree} schon so groß wird, dass kaum noch belastbare
        Aussagen getroffen werden können. Aus diesem Grund wurde darauf verzichtet, den Wert für \(\alpha = \SI{-5}{\degree}\)
        aufzutragen.
        \begin{figure}[h]
            \centering
            \includesvg[inkscapelatex=false, width=.7\linewidth]{scidavis/kennlinienfeldDuese.svg}
            \caption[Anlagenkennlinenfeld Duese]{Anlagenkennlinienfeld eines Diffusors unter Variation des Öffnungswinkels.}
            \label{fig:anlagenkennlinienfeld Duese}
        \end{figure}
        \begin{figure}[h]
            \centering
            \includesvg[inkscapelatex=false, width=.7\linewidth]{scidavis/zetasDuese.svg}
            \caption[Druckverlustbeiwert als Funktion des Verjüngungswinkels eine Düse]{Verlauf des Druckverlustbeiwertes \(\zeta\) einer Düse als Funktion ihres Verjüngungswinkels \(\alpha\).}
            \label{fig:zetas Duese}
        \end{figure}